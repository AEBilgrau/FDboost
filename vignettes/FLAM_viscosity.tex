
\documentclass{article}\usepackage[]{graphicx}\usepackage[]{color}
%% maxwidth is the original width if it is less than linewidth
%% otherwise use linewidth (to make sure the graphics do not exceed the margin)
\makeatletter
\def\maxwidth{ %
  \ifdim\Gin@nat@width>\linewidth
    \linewidth
  \else
    \Gin@nat@width
  \fi
}
\makeatother

\definecolor{fgcolor}{rgb}{0.345, 0.345, 0.345}
\newcommand{\hlnum}[1]{\textcolor[rgb]{0.686,0.059,0.569}{#1}}%
\newcommand{\hlstr}[1]{\textcolor[rgb]{0.192,0.494,0.8}{#1}}%
\newcommand{\hlcom}[1]{\textcolor[rgb]{0.678,0.584,0.686}{\textit{#1}}}%
\newcommand{\hlopt}[1]{\textcolor[rgb]{0,0,0}{#1}}%
\newcommand{\hlstd}[1]{\textcolor[rgb]{0.345,0.345,0.345}{#1}}%
\newcommand{\hlkwa}[1]{\textcolor[rgb]{0.161,0.373,0.58}{\textbf{#1}}}%
\newcommand{\hlkwb}[1]{\textcolor[rgb]{0.69,0.353,0.396}{#1}}%
\newcommand{\hlkwc}[1]{\textcolor[rgb]{0.333,0.667,0.333}{#1}}%
\newcommand{\hlkwd}[1]{\textcolor[rgb]{0.737,0.353,0.396}{\textbf{#1}}}%
\let\hlipl\hlkwb

\usepackage{framed}
\makeatletter
\newenvironment{kframe}{%
 \def\at@end@of@kframe{}%
 \ifinner\ifhmode%
  \def\at@end@of@kframe{\end{minipage}}%
  \begin{minipage}{\columnwidth}%
 \fi\fi%
 \def\FrameCommand##1{\hskip\@totalleftmargin \hskip-\fboxsep
 \colorbox{shadecolor}{##1}\hskip-\fboxsep
     % There is no \\@totalrightmargin, so:
     \hskip-\linewidth \hskip-\@totalleftmargin \hskip\columnwidth}%
 \MakeFramed {\advance\hsize-\width
   \@totalleftmargin\z@ \linewidth\hsize
   \@setminipage}}%
 {\par\unskip\endMakeFramed%
 \at@end@of@kframe}
\makeatother

\definecolor{shadecolor}{rgb}{.97, .97, .97}
\definecolor{messagecolor}{rgb}{0, 0, 0}
\definecolor{warningcolor}{rgb}{1, 0, 1}
\definecolor{errorcolor}{rgb}{1, 0, 0}
\newenvironment{knitrout}{}{} % an empty environment to be redefined in TeX

\usepackage{alltt}
\usepackage{amstext}
\usepackage{amsfonts}
\usepackage{hyperref}
\usepackage[round]{natbib}
\usepackage{hyperref}
\usepackage{graphicx}
\usepackage{rotating}
\usepackage{authblk}
\usepackage[left=25mm, right=25mm, top=20mm, bottom=20mm]{geometry}
%\usepackage[nolists]{endfloat}

%\VignetteEngine{knitr::knitr}
%\VignetteDepends{FDboost, fda, fields, maps, mapdata}
%\VignetteIndexEntry{FDboost FLAM viscosity}

\newcommand{\Rpackage}[1]{{\normalfont\fontseries{b}\selectfont #1}}
\newcommand{\Robject}[1]{\texttt{#1}}
\newcommand{\Rclass}[1]{\textit{#1}}
\newcommand{\Rcmd}[1]{\texttt{#1}}
\newcommand{\Roperator}[1]{\texttt{#1}}
\newcommand{\Rarg}[1]{\texttt{#1}}
\newcommand{\Rlevel}[1]{\texttt{#1}}

\newcommand{\RR}{\textsf{R}}
\renewcommand{\S}{\textsf{S}}
\newcommand{\df}{\mbox{df}}

\RequirePackage[T1]{fontenc}
\RequirePackage{graphicx,ae,fancyvrb}
\IfFileExists{upquote.sty}{\RequirePackage{upquote}}{}
\usepackage{relsize}

\renewcommand{\baselinestretch}{1}
\setlength\parindent{0pt}


\hypersetup{%
  pdftitle = {FLAM canada},
  pdfsubject = {package vignette},
  pdfauthor = {Sarah Brockhaus},
%% change colorlinks to false for pretty printing
  colorlinks = {true},
  linkcolor = {blue},
  citecolor = {blue},
  urlcolor = {red},
  hyperindex = {true},
  linktocpage = {true},
}
\IfFileExists{upquote.sty}{\usepackage{upquote}}{}
\begin{document}

\setkeys{Gin}{width=\textwidth}

\title{Canadian climate: function-on-function regression}
\author{Sarah Brockhaus 
\thanks{E-mail: sarah.brockhaus@stat.uni-muenchen.de}}
\affil{\textit{Institut f\"ur Statistik, \\
Ludwig-Maximilians-Universit\"at M\"unchen, \\ 
Ludwigstra{\ss}e 33, D-80539 M\"unchen, Germany.}}
\date{}
\maketitle

The results of this vignette together with more explanations can be found in Brockhaus et al. (2015). 



\section{Descriptive analysis}

Load FDboost package and write useful functions for plotting.


Load data and choose the time-interval.
\begin{knitrout}
\definecolor{shadecolor}{rgb}{0.969, 0.969, 0.969}\color{fgcolor}\begin{kframe}
\begin{alltt}
\hlcom{# load("viscosity.RData")}
\hlkwd{data}\hlstd{(viscosity)}
\hlkwd{str}\hlstd{(viscosity)}
\end{alltt}
\begin{verbatim}
List of 7
 $ visAll : AsIs [1:64, 1:132] 41.5 25.2 63.7 35.6 17.8 12.3 38.6   22 18.2   36 ...
 $ timeAll: num [1:132] 11 13 15 17 19 21 23 25 27 29 ...
 $ T_C    : Factor w/ 2 levels "low","high": 1 1 2 2 2 2 1 1 1 1 ...
 $ T_A    : Factor w/ 2 levels "low","high": 1 1 1 1 1 1 1 1 1 1 ...
 $ T_B    : Factor w/ 2 levels "low","high": 1 1 1 1 1 1 1 1 2 2 ...
 $ rspeed : Factor w/ 2 levels "low","high": 1 2 1 2 1 2 2 1 2 1 ...
 $ mflow  : Factor w/ 2 levels "low","high": 2 1 1 2 1 2 1 2 2 1 ...
\end{verbatim}
\begin{alltt}
\hlcom{## set time-interval that should be modeled}
\hlstd{interval} \hlkwb{<-} \hlstr{"509"}

\hlcom{## model time until "interval"}
\hlstd{end} \hlkwb{<-} \hlkwd{which}\hlstd{(viscosity}\hlopt{$}\hlstd{timeAll}\hlopt{==}\hlkwd{as.numeric}\hlstd{(interval))}
\hlstd{viscosity}\hlopt{$}\hlstd{vis} \hlkwb{<-} \hlkwd{log}\hlstd{(viscosity}\hlopt{$}\hlstd{visAll[,}\hlnum{1}\hlopt{:}\hlstd{end])}
\hlstd{viscosity}\hlopt{$}\hlstd{time} \hlkwb{<-} \hlstd{viscosity}\hlopt{$}\hlstd{timeAll[}\hlnum{1}\hlopt{:}\hlstd{end]}

\hlcom{## set up interactions by hand}
\hlstd{vars} \hlkwb{<-} \hlkwd{c}\hlstd{(}\hlstr{"T_C"}\hlstd{,} \hlstr{"T_A"}\hlstd{,} \hlstr{"T_B"}\hlstd{,} \hlstr{"rspeed"}\hlstd{,} \hlstr{"mflow"}\hlstd{)}
\hlkwa{for}\hlstd{(v} \hlkwa{in} \hlnum{1}\hlopt{:}\hlkwd{length}\hlstd{(vars))\{}
  \hlkwa{for}\hlstd{(w} \hlkwa{in} \hlstd{v}\hlopt{:}\hlkwd{length}\hlstd{(vars))}
  \hlstd{viscosity[[}\hlkwd{paste}\hlstd{(vars[v], vars[w],} \hlkwc{sep}\hlstd{=}\hlstr{"_"}\hlstd{)]]} \hlkwb{<-} \hlkwd{factor}\hlstd{(}
    \hlstd{(viscosity[[vars[v]]]}\hlopt{:}\hlstd{viscosity[[vars[w]]]}\hlopt{==}\hlstr{"high:high"}\hlstd{)}\hlopt{*}\hlnum{1}\hlstd{)}
\hlstd{\}}

\hlcom{#str(viscosity)}
\hlkwd{names}\hlstd{(viscosity)}
\end{alltt}
\begin{verbatim}
 [1] "visAll"        "timeAll"       "T_C"           "T_A"          
 [5] "T_B"           "rspeed"        "mflow"         "vis"          
 [9] "time"          "T_C_T_C"       "T_C_T_A"       "T_C_T_B"      
[13] "T_C_rspeed"    "T_C_mflow"     "T_A_T_A"       "T_A_T_B"      
[17] "T_A_rspeed"    "T_A_mflow"     "T_B_T_B"       "T_B_rspeed"   
[21] "T_B_mflow"     "rspeed_rspeed" "rspeed_mflow"  "mflow_mflow"  
\end{verbatim}
\end{kframe}
\end{knitrout}

Plot the data

\begin{knitrout}
\definecolor{shadecolor}{rgb}{0.969, 0.969, 0.969}\color{fgcolor}\begin{kframe}
\begin{alltt}
\hlkwd{par}\hlstd{(}\hlkwc{mfrow}\hlstd{=}\hlkwd{c}\hlstd{(}\hlnum{1}\hlstd{,}\hlnum{1}\hlstd{),} \hlkwc{mar}\hlstd{=}\hlkwd{c}\hlstd{(}\hlnum{3}\hlstd{,} \hlnum{3}\hlstd{,} \hlnum{1}\hlstd{,} \hlnum{2}\hlstd{))}\hlcom{#, cex=1.5)}
\hlstd{mycol} \hlkwb{<-} \hlkwd{gray}\hlstd{(}\hlkwd{seq}\hlstd{(}\hlnum{0}\hlstd{,} \hlnum{0.8}\hlstd{,} \hlkwc{l}\hlstd{=}\hlnum{4}\hlstd{),} \hlkwc{alpha}\hlstd{=}\hlnum{0.8}\hlstd{)[}\hlkwd{c}\hlstd{(}\hlnum{1}\hlstd{,}\hlnum{3}\hlstd{,}\hlnum{2}\hlstd{,}\hlnum{4}\hlstd{)]}
\hlstd{int_T_CA} \hlkwb{<-} \hlkwd{with}\hlstd{(viscosity,} \hlkwd{paste}\hlstd{(T_C,}\hlstr{"-"}\hlstd{, T_A,} \hlkwc{sep}\hlstd{=}\hlstr{""}\hlstd{))}
\hlkwd{with}\hlstd{(viscosity,} \hlkwd{funplotLogscale}\hlstd{(time, vis,}
                                \hlkwc{col}\hlstd{=}\hlkwd{getCol2}\hlstd{(int_T_CA,} \hlkwc{cols}\hlstd{=mycol[}\hlnum{4}\hlopt{:}\hlnum{1}\hlstd{])))}
\hlkwd{legend}\hlstd{(}\hlstr{"bottomright"}\hlstd{,} \hlkwc{fill}\hlstd{=mycol,}
       \hlkwc{legend}\hlstd{=}\hlkwd{c}\hlstd{(}\hlstr{"T_C low, T_A low"}\hlstd{,} \hlstr{"T_C low, T_A high"}\hlstd{,}
                \hlstr{"T_C high, T_A low"}\hlstd{,} \hlstr{"T_C high, T_A high"}\hlstd{),} \hlkwc{cex} \hlstd{=} \hlnum{0.8}\hlstd{)}
\end{alltt}
\end{kframe}\begin{figure}

{\centering \includegraphics[width=\maxwidth]{figure/plot-data-1} 

}

\caption[Viscostiy over time with temperature of tools ($T_C$) and temerature of resin ($T_A$) color coded]{Viscostiy over time with temperature of tools ($T_C$) and temerature of resin ($T_A$) color coded.}\label{fig:plot-data}
\end{figure}


\end{knitrout}

%%%%%%%%%%%%%%%%%%%%%%%%%%%%%%%%%%%%%%%%%%%%%%%%%%%%%%%%%%%%%%%%%%%%%%%%%%%%%%%%%%%%%%%%%%%%

\newpage

\section{Model with all main effects and interactions of first order}

Fit model with all main effects and interactions. 

\begin{knitrout}
\definecolor{shadecolor}{rgb}{0.969, 0.969, 0.969}\color{fgcolor}\begin{kframe}
\begin{alltt}
\hlkwd{set.seed}\hlstd{(}\hlnum{1911}\hlstd{)}
\hlstd{modAll} \hlkwb{<-} \hlkwd{FDboost}\hlstd{(vis} \hlopt{~} \hlnum{1}
                  \hlopt{+} \hlkwd{bols}\hlstd{(T_C)} \hlcom{# main effects}
                  \hlopt{+} \hlkwd{bols}\hlstd{(T_A)}
                  \hlopt{+} \hlkwd{bols}\hlstd{(T_B)}
                  \hlopt{+} \hlkwd{bols}\hlstd{(rspeed)}
                  \hlopt{+} \hlkwd{bols}\hlstd{(mflow)}
                  \hlopt{+} \hlkwd{bols}\hlstd{(T_C_T_A)} \hlcom{# interactions T_WZ}
                  \hlopt{+} \hlkwd{bols}\hlstd{(T_C_T_B)}
                  \hlopt{+} \hlkwd{bols}\hlstd{(T_C_rspeed)}
                  \hlopt{+} \hlkwd{bols}\hlstd{(T_C_mflow)}
                  \hlopt{+} \hlkwd{bols}\hlstd{(T_A_T_B)} \hlcom{# interactions T_A}
                  \hlopt{+} \hlkwd{bols}\hlstd{(T_A_rspeed)}
                  \hlopt{+} \hlkwd{bols}\hlstd{(T_A_mflow)}
                  \hlopt{+} \hlkwd{bols}\hlstd{(T_B_rspeed)} \hlcom{# interactions T_B}
                  \hlopt{+} \hlkwd{bols}\hlstd{(T_B_mflow)}
                  \hlopt{+} \hlkwd{bols}\hlstd{(rspeed_mflow),} \hlcom{# interactions rspeed}
                  \hlkwc{timeformula}\hlstd{=}\hlopt{~}\hlkwd{bbs}\hlstd{(time,} \hlkwc{lambda}\hlstd{=}\hlnum{100}\hlstd{),}
                  \hlkwc{numInt}\hlstd{=}\hlstr{"Riemann"}\hlstd{,} \hlkwc{family}\hlstd{=}\hlkwd{QuantReg}\hlstd{(),}
                  \hlkwc{offset}\hlstd{=}\hlkwa{NULL}\hlstd{,} \hlkwc{offset_control} \hlstd{=} \hlkwd{o_control}\hlstd{(}\hlkwc{k_min} \hlstd{=} \hlnum{10}\hlstd{),}
                  \hlkwc{data}\hlstd{=viscosity,} \hlkwc{check0}\hlstd{=}\hlnum{FALSE}\hlstd{,}
                  \hlkwc{control}\hlstd{=}\hlkwd{boost_control}\hlstd{(}\hlkwc{mstop} \hlstd{=} \hlnum{100}\hlstd{,} \hlkwc{nu} \hlstd{=} \hlnum{0.2}\hlstd{))}
\end{alltt}
\end{kframe}
\end{knitrout}

Get optimal stopping iteration using bootstrap over curves.
\begin{knitrout}
\definecolor{shadecolor}{rgb}{0.969, 0.969, 0.969}\color{fgcolor}\begin{kframe}
\begin{alltt}
\hlkwd{set.seed}\hlstd{(}\hlnum{1911}\hlstd{)}
\hlstd{folds} \hlkwb{<-} \hlkwd{cv}\hlstd{(}\hlkwc{weights}\hlstd{=}\hlkwd{rep}\hlstd{(}\hlnum{1}\hlstd{, modAll}\hlopt{$}\hlstd{ydim[}\hlnum{1}\hlstd{]),} \hlkwc{type}\hlstd{=}\hlstr{"bootstrap"}\hlstd{,} \hlkwc{B}\hlstd{=}\hlnum{10}\hlstd{)}
\hlstd{cvmAll} \hlkwb{<-} \hlkwd{suppressWarnings}\hlstd{(}\hlkwd{validateFDboost}\hlstd{(modAll,} \hlkwc{folds} \hlstd{= folds,}
                                  \hlkwc{getCoefCV}\hlstd{=}\hlnum{FALSE}\hlstd{,}
                                  \hlkwc{grid}\hlstd{=}\hlkwd{seq}\hlstd{(}\hlnum{10}\hlstd{,} \hlnum{500}\hlstd{,} \hlkwc{by}\hlstd{=}\hlnum{10}\hlstd{),} \hlkwc{mc.cores}\hlstd{=}\hlnum{10}\hlstd{))}
\hlkwd{mstop}\hlstd{(cvmAll)} \hlcom{# 180}
\hlcom{# modAll <- modAll[mstop(cvmAll)]}
\hlcom{# summary(modAll)}
\hlcom{# cvmAll}
\end{alltt}
\end{kframe}
\end{knitrout}

% \begin{figure}
% \begin{center}
% <<cv-plot-complete, echo = TRUE, fig = TRUE, width=10, heigth=5, eval=FALSE>>=
%  par(mfrow=c(1,2))
%  plot(cvmAll)
%@
% \end{center}
% \caption{Optimal stopping iteration for model with all main effects and interactions of first order.}
% \end{figure}


Do model selection using stability selection.
\begin{knitrout}
\definecolor{shadecolor}{rgb}{0.969, 0.969, 0.969}\color{fgcolor}\begin{kframe}
\begin{alltt}
\hlkwd{set.seed}\hlstd{(}\hlnum{1911}\hlstd{)}
\hlstd{folds} \hlkwb{<-} \hlkwd{cvMa}\hlstd{(}\hlkwc{ydim}\hlstd{=modAll}\hlopt{$}\hlstd{ydim,} \hlkwc{weights}\hlstd{=}\hlkwd{model.weights}\hlstd{(modAll),}
              \hlkwc{type} \hlstd{=} \hlstr{"subsampling"}\hlstd{,} \hlkwc{B} \hlstd{=} \hlnum{50}\hlstd{)}

\hlkwd{stabsel_parameters}\hlstd{(}\hlkwc{q}\hlstd{=}\hlnum{5}\hlstd{,} \hlkwc{PFER}\hlstd{=}\hlnum{2}\hlstd{,} \hlkwc{p}\hlstd{=}\hlnum{16}\hlstd{,} \hlkwc{sampling.type} \hlstd{=} \hlstr{"SS"}\hlstd{)}
\hlstd{sel1} \hlkwb{<-} \hlkwd{stabsel}\hlstd{(modAll,} \hlkwc{q}\hlstd{=}\hlnum{5}\hlstd{,} \hlkwc{PFER}\hlstd{=}\hlnum{2}\hlstd{,} \hlkwc{folds}\hlstd{=folds,} \hlkwc{grid}\hlstd{=}\hlnum{1}\hlopt{:}\hlnum{100}\hlstd{,}
                \hlkwc{sampling.type}\hlstd{=}\hlstr{"SS"}\hlstd{,} \hlkwc{mc.cores}\hlstd{=}\hlnum{10}\hlstd{)}
\hlstd{sel1}
\hlcom{# selects effects T_C, T_A, T_C_T_A}
\end{alltt}
\end{kframe}
\end{knitrout}
The effects $T_A$, $T_C$ and their interaction are selected into the model. 


% <<stabsel2, echo = TRUE>>=
% set.seed(1911)
% folds <- cvMa(ydim=modAll$ydim, weights=model.weights(modAll), 
%               type = "subsampling", B = 50)
% 
% stabsel_parameters(q=3, PFER=1, p=16, sampling.type = "SS")
% sel2 <- stabsel(modAll, q=3, PFER=1, folds=folds, grid=1:100, 
%                 sampling.type="SS", mc.cores=10)
% sel2
% # selects effects of T_C, T_A
% @




%%%%%%%%%%%%%%%%%%%%%%%%%%%%%%%%%%%%%%%%%%%%%%%%%%%%%%%%%%%%%%%%%%%%%%%%%%%%%%%%%%%%%%%%%%%%

\newpage

\section{Model with selected effects}

Estimate the model containig only the selected effects $T_C$, $T_A$, and their interaction.
\begin{knitrout}
\definecolor{shadecolor}{rgb}{0.969, 0.969, 0.969}\color{fgcolor}\begin{kframe}
\begin{alltt}
\hlkwd{set.seed}\hlstd{(}\hlnum{1911}\hlstd{)}
\hlstd{mod1} \hlkwb{<-} \hlkwd{FDboost}\hlstd{(vis} \hlopt{~} \hlnum{1} \hlopt{+} \hlkwd{bols}\hlstd{(T_C)} \hlopt{+} \hlkwd{bols}\hlstd{(T_A)} \hlopt{+} \hlkwd{bols}\hlstd{(T_C_T_A),}
                \hlkwc{timeformula} \hlstd{=} \hlopt{~}\hlkwd{bbs}\hlstd{(time,} \hlkwc{lambda} \hlstd{=} \hlnum{100}\hlstd{),}
                \hlkwc{numInt} \hlstd{=} \hlstr{"Riemann"}\hlstd{,} \hlkwc{family} \hlstd{=} \hlkwd{QuantReg}\hlstd{(),} \hlkwc{check0} \hlstd{=} \hlnum{FALSE}\hlstd{,}
                \hlkwc{offset} \hlstd{=} \hlkwa{NULL}\hlstd{,} \hlkwc{offset_control} \hlstd{=} \hlkwd{o_control}\hlstd{(}\hlkwc{k_min} \hlstd{=} \hlnum{10}\hlstd{),}
                \hlkwc{data} \hlstd{= viscosity,} \hlkwc{control} \hlstd{=} \hlkwd{boost_control}\hlstd{(}\hlkwc{mstop} \hlstd{=} \hlnum{200}\hlstd{,} \hlkwc{nu} \hlstd{=} \hlnum{0.2}\hlstd{))}
\end{alltt}
\end{kframe}
\end{knitrout}

\begin{knitrout}
\definecolor{shadecolor}{rgb}{0.969, 0.969, 0.969}\color{fgcolor}\begin{kframe}
\begin{alltt}
\hlstd{mod1} \hlkwb{<-} \hlstd{mod1[}\hlnum{430}\hlstd{]}
\end{alltt}
\end{kframe}
\end{knitrout}

Find the optimal stopping iteration.
\begin{knitrout}
\definecolor{shadecolor}{rgb}{0.969, 0.969, 0.969}\color{fgcolor}\begin{kframe}
\begin{alltt}
\hlkwd{set.seed}\hlstd{(}\hlnum{1911}\hlstd{)}
\hlstd{folds} \hlkwb{<-} \hlkwd{cv}\hlstd{(}\hlkwc{weights} \hlstd{=} \hlkwd{rep}\hlstd{(}\hlnum{1}\hlstd{, mod1}\hlopt{$}\hlstd{ydim[}\hlnum{1}\hlstd{]),} \hlkwc{type} \hlstd{=} \hlstr{"bootstrap"}\hlstd{,} \hlkwc{B} \hlstd{=} \hlnum{10}\hlstd{)}
\hlstd{cvm1} \hlkwb{<-} \hlkwd{validateFDboost}\hlstd{(mod1,} \hlkwc{folds} \hlstd{= folds,} \hlkwc{getCoefCV} \hlstd{=} \hlnum{FALSE}\hlstd{,}
                        \hlkwc{grid} \hlstd{=} \hlkwd{seq}\hlstd{(}\hlnum{10}\hlstd{,} \hlnum{500}\hlstd{,} \hlkwc{by} \hlstd{=} \hlnum{10}\hlstd{),} \hlkwc{mc.cores} \hlstd{=} \hlnum{10}\hlstd{)}
\hlkwd{mstop}\hlstd{(cvm1)} \hlcom{# 430}
\hlstd{mod1} \hlkwb{<-} \hlstd{mod1[}\hlkwd{mstop}\hlstd{(cvm1)]}
\hlcom{# summary(mod1)}
\end{alltt}
\end{kframe}
\end{knitrout}

% \begin{figure}
% \begin{center}
% <<cv-plot-selected-model, echo = TRUE, fig = TRUE, width=10, heigth=5, eval=FALSE>>=
%  par(mfrow=c(1,2))
%   plot(cvm1) 
% @
% \end{center}
% \caption{Optimal stopping iteration for model with selected effects.}
% \end{figure}


Center all coefficient functions at each timepoint, yielding the following model: 
\[\mbox{median} \{  \log( \mbox{vis}_i(t)) | x_i  \}  = \beta_0(t) + T_{Ai} \beta_A(t)  + T_{Ci} \beta_C(t) +  T_{ACi} \beta_{AC}(t), \]
where $\mbox{vis}_i(t)$ is the viscosity of observation $i$ at time $t$, $T_{Ai}$ and $T_{Ci}$ are the temperatures of resin and of tools, respectively, each coded as -1 for the lower and 1 for the higher temperature. The interaction $T_{ACi}$ is 1 if both temperatures are in the higher category and -1 otherwise.
\begin{knitrout}
\definecolor{shadecolor}{rgb}{0.969, 0.969, 0.969}\color{fgcolor}\begin{kframe}
\begin{alltt}
\hlcom{# set up dataframe containing systematically all variable combinations}
\hlstd{newdata} \hlkwb{<-} \hlkwd{list}\hlstd{(}\hlkwc{T_C}\hlstd{=}\hlkwd{factor}\hlstd{(}\hlkwd{c}\hlstd{(}\hlnum{1}\hlstd{,}\hlnum{1}\hlstd{,}\hlnum{2}\hlstd{,}\hlnum{2}\hlstd{),} \hlkwc{levels}\hlstd{=}\hlnum{1}\hlopt{:}\hlnum{2}\hlstd{,} \hlkwc{labels}\hlstd{=}\hlkwd{c}\hlstd{(}\hlstr{"low"}\hlstd{,}\hlstr{"high"}\hlstd{)) ,}
             \hlkwc{T_A}\hlstd{=}\hlkwd{factor}\hlstd{(}\hlkwd{c}\hlstd{(}\hlnum{1}\hlstd{,} \hlnum{2}\hlstd{,} \hlnum{1}\hlstd{,} \hlnum{2}\hlstd{),} \hlkwc{levels}\hlstd{=}\hlnum{1}\hlopt{:}\hlnum{2}\hlstd{,} \hlkwc{labels}\hlstd{=}\hlkwd{c}\hlstd{(}\hlstr{"low"}\hlstd{,}\hlstr{"high"}\hlstd{)),}
             \hlkwc{T_C_T_A}\hlstd{=}\hlkwd{factor}\hlstd{(}\hlkwd{c}\hlstd{(}\hlnum{1}\hlstd{,} \hlnum{1}\hlstd{,} \hlnum{1}\hlstd{,} \hlnum{2}\hlstd{)),} \hlkwc{time}\hlstd{=mod1}\hlopt{$}\hlstd{yind)}
\hlstd{intercept} \hlkwb{<-} \hlnum{0}

\hlcom{## effect of T_C}
\hlstd{pred2} \hlkwb{<-} \hlkwd{predict}\hlstd{(mod1,} \hlkwc{which}\hlstd{=}\hlnum{2}\hlstd{,} \hlkwc{newdata}\hlstd{=newdata)}
\hlstd{intercept} \hlkwb{<-} \hlstd{intercept} \hlopt{+} \hlkwd{colMeans}\hlstd{(pred2)}
\hlstd{pred2} \hlkwb{<-} \hlkwd{t}\hlstd{(}\hlkwd{t}\hlstd{(pred2)}\hlopt{-}\hlstd{intercept)}

\hlcom{## effect of T_A}
\hlstd{pred3} \hlkwb{<-} \hlkwd{predict}\hlstd{(mod1,} \hlkwc{which}\hlstd{=}\hlnum{3}\hlstd{,} \hlkwc{newdata}\hlstd{=newdata)}
\hlstd{intercept} \hlkwb{<-} \hlstd{intercept} \hlopt{+} \hlkwd{colMeans}\hlstd{(pred3)}
\hlstd{pred3} \hlkwb{<-} \hlkwd{t}\hlstd{(}\hlkwd{t}\hlstd{(pred3)}\hlopt{-}\hlkwd{colMeans}\hlstd{(pred3))}

\hlcom{## interaction effect T_C_T_A}
\hlstd{pred4} \hlkwb{<-} \hlkwd{predict}\hlstd{(mod1,} \hlkwc{which}\hlstd{=}\hlnum{4}\hlstd{,} \hlkwc{newdata}\hlstd{=newdata)}
\hlstd{intercept} \hlkwb{<-} \hlstd{intercept} \hlopt{+} \hlkwd{colMeans}\hlstd{(pred4[}\hlnum{3}\hlopt{:}\hlnum{4}\hlstd{,])}
\hlstd{pred4} \hlkwb{<-} \hlkwd{t}\hlstd{(}\hlkwd{t}\hlstd{(pred4)}\hlopt{-}\hlkwd{colMeans}\hlstd{(pred4[}\hlnum{3}\hlopt{:}\hlnum{4}\hlstd{,]))}

\hlcom{# offset+intercept }
\hlstd{smoothIntercept} \hlkwb{<-} \hlstd{mod1}\hlopt{$}\hlkwd{predictOffset}\hlstd{(newdata}\hlopt{$}\hlstd{time)} \hlopt{+} \hlstd{intercept}
\end{alltt}
\end{kframe}
\end{knitrout}

Plot the centered coefficient functions.
\begin{knitrout}
\definecolor{shadecolor}{rgb}{0.969, 0.969, 0.969}\color{fgcolor}\begin{kframe}
\begin{alltt}
\hlkwd{lines}\hlstd{(mod1}\hlopt{$}\hlstd{yind, pred4[}\hlnum{4}\hlstd{,],} \hlkwc{col}\hlstd{=mycol[}\hlnum{3}\hlstd{],} \hlkwc{lty}\hlstd{=}\hlnum{4}\hlstd{,} \hlkwc{lwd}\hlstd{=}\hlnum{2}\hlstd{)}
\hlkwd{legend}\hlstd{(}\hlstr{"topright"}\hlstd{,} \hlkwc{lty}\hlstd{=}\hlnum{2}\hlopt{:}\hlnum{4}\hlstd{,} \hlkwc{lwd}\hlstd{=}\hlnum{2}\hlstd{,} \hlkwc{col}\hlstd{=mycol,}
       \hlkwc{legend}\hlstd{=}\hlkwd{c}\hlstd{(}\hlstr{"effect T_C high"}\hlstd{,} \hlstr{"effect T_A high"}\hlstd{,} \hlstr{"effect T_C, T_A high"}\hlstd{))}
\end{alltt}
\end{kframe}\begin{figure}
\includegraphics[width=\maxwidth]{figure/plot-selected-model-1} \caption[Viscosity over time and estimated coefficient functions]{Viscosity over time and estimated coefficient functions. On the left hand side the viscosity measures are plotted over time with temperature of tools ($T_C$) and temperature of resin ($T_A$) color-coded. On the right hand side the estimated coefficient functions are plotted.}\label{fig:plot-selected-model}
\end{figure}


\end{knitrout}

\section*{References}
\begin{itemize}
\item[] Brockhaus S, Scheipl, F., Hothor, T., and Greven, S. (2015), The functional linear array model, 
        \textit{Statistical Modelling}, 15(3), 279--300. 
\end{itemize}


\end{document}
